\documentclass[12pt]{article}
\usepackage[top=1in,bottom=1in,left=0.75in,right=0.75in,centering]{geometry}
\usepackage{fancyhdr}
\usepackage{epsfig}
\usepackage[pdfborder={0 0 0}]{hyperref}
\usepackage{palatino}
\usepackage{wrapfig}
\usepackage{lastpage}
\usepackage{color}
\usepackage{ifthen}
\usepackage[table]{xcolor}
\usepackage{graphicx,type1cm,eso-pic,color}
\usepackage{hyperref}
\usepackage{amsmath}
\usepackage{wasysym}

\def\course{CS 3120: Discrete Math and Theory II}
\def\homework{What is a computer? Proof Techniques}
\def\semester{Fall 2023}

\newboolean{solution}
\setboolean{solution}{false}

% add watermark if it's a solution exam
% see http://jeanmartina.blogspot.com/2008/07/latex-goodie-how-to-watermark-things-in.html
\makeatletter
\AddToShipoutPicture{%
\setlength{\@tempdimb}{.5\paperwidth}%
\setlength{\@tempdimc}{.5\paperheight}%
\setlength{\unitlength}{1pt}%
\put(\strip@pt\@tempdimb,\strip@pt\@tempdimc){%
\ifthenelse{\boolean{solution}}{
\makebox(0,0){\rotatebox{45}{\textcolor[gray]{0.95}%
{\fontsize{5cm}{3cm}\selectfont{\textsf{Solution}}}}}%
}{}
}}
\makeatother

\pagestyle{fancy}

\fancyhf{}
\lhead{\course}
\chead{Page \thepage\ of \pageref{LastPage}}
\rhead{\semester}
%\cfoot{\Large (the bubble footer is automatically inserted into this space)}

\setlength{\headheight}{14.5pt}

\newenvironment{itemlist}{
\begin{itemize}
\setlength{\itemsep}{0pt}
\setlength{\parskip}{0pt}}
{\end{itemize}}

\newenvironment{numlist}{
\begin{enumerate}
\setlength{\itemsep}{0pt}
\setlength{\parskip}{0pt}}
{\end{enumerate}}

\newcounter{pagenum}
\setcounter{pagenum}{1}
\newcommand{\pageheader}[1]{
\clearpage\vspace*{-0.4in}\noindent{\large\bf{Page \arabic{pagenum}: {#1}}}
\addtocounter{pagenum}{1}
\cfoot{}
}

\newcounter{quesnum}
\setcounter{quesnum}{1}
\newcommand{\question}[2][??]{
\begin{list}{\labelitemi}{\leftmargin=2em}
\item [\arabic{quesnum}.] {} {#2}
\end{list}
\addtocounter{quesnum}{1}
}


\definecolor{red}{rgb}{1.0,0.0,0.0}
\newcommand{\answer}[2][??]{
\ifthenelse{\boolean{solution}}{
\color{red} #2 \color{black}}
{\vspace*{#1}}
}

\definecolor{blue}{rgb}{0.0,0.0,1.0}

\begin{document}

\section*{\homework}


\question[3]{
Suppose you have an alphabet $\Sigma$ that has a specific length $m$ (i.e., $|\Sigma|=m$). How long is the alphabet $\Sigma^n$? In other words, what is the value of $|\Sigma^n|$? Write out your formula and provide 1 or 2 sentences justifying the formula.
}

\vspace{12pt}

\question[3]{
If the set $A$ has $a$ elements, and the set $B$ has $b$ elements, how many elements are in $A \times B$? Explain your answer in a sentence or two.
}

\vspace{12pt}

\question[3]{
Suppose that $\Sigma = \{0,1\}$. Given the following intuitive descriptions of alphabets, provide a description of each of the same alphabets using formal, set notation.

\begin{itemize}
	\item Strings that begin with $011$ or begin with $100$.
	\item Strings that contain four $1$s in a row at least once.
	\item Strings that have two $1$s in a row somewhere in the string. However, the two ones are NOT the first two characters or the last two characters.
\end{itemize}
}


\question[3]{
Suppose I build a new computing machine that can be programmed to recognize a lot of different functions! It is called the \emph{Flogrammable Device}. In order to program this machine, you can type out your program on a \emph{tape}, but this type can only hold ten unique charactes. Each character is from the alphabet $\Sigma=\{a,b,c,0,1\}$ and any combination of these 10 characters is a valid program. You CANNOT have fewer than 10 characters or the code will not compile.\\
\\
Now suppose that you read online that somehow, there are only 10,000,000 important functions that need to be computed by the \emph{Flogrammable Device} for it to cover all important functionality. Can we program each of the 10,000,000 functions on this machine? How do you know or not?
}

\vspace{12pt}

\question[3]{
\textbf{Proof Practice:} Prove that every graph $G = (V,E)$ such that $|V| \geq 2$ contains two nodes with equal degrees.
}



\end{document}
