\documentclass[12pt]{article}
\usepackage[top=1in,bottom=1in,left=0.75in,right=0.75in,centering]{geometry}
\usepackage{fancyhdr}
\usepackage{epsfig}
\usepackage[pdfborder={0 0 0}]{hyperref}
\usepackage{palatino}
\usepackage{wrapfig}
\usepackage{lastpage}
\usepackage{color}
\usepackage{ifthen}
\usepackage[table]{xcolor}
\usepackage{graphicx,type1cm,eso-pic,color}
\usepackage{hyperref}
\usepackage{amsmath}
\usepackage{wasysym}
\usepackage{amsfonts}

\def\course{CS 3120: Discrete Math and Theory II}
\def\homework{Set Cardinality}
\def\semester{Spring 2025}

\newboolean{solution}
\setboolean{solution}{false}

% add watermark if it's a solution exam
% see http://jeanmartina.blogspot.com/2008/07/latex-goodie-how-to-watermark-things-in.html
\makeatletter
\AddToShipoutPicture{%
\setlength{\@tempdimb}{.5\paperwidth}%
\setlength{\@tempdimc}{.5\paperheight}%
\setlength{\unitlength}{1pt}%
\put(\strip@pt\@tempdimb,\strip@pt\@tempdimc){%
\ifthenelse{\boolean{solution}}{
\makebox(0,0){\rotatebox{45}{\textcolor[gray]{0.95}%
{\fontsize{5cm}{3cm}\selectfont{\textsf{Solution}}}}}%
}{}
}}
\makeatother

\pagestyle{fancy}

\fancyhf{}
\lhead{\course}
\chead{Page \thepage\ of \pageref{LastPage}}
\rhead{\semester}
%\cfoot{\Large (the bubble footer is automatically inserted into this space)}

\setlength{\headheight}{14.5pt}

\newenvironment{itemlist}{
\begin{itemize}
\setlength{\itemsep}{0pt}
\setlength{\parskip}{0pt}}
{\end{itemize}}

\newenvironment{numlist}{
\begin{enumerate}
\setlength{\itemsep}{0pt}
\setlength{\parskip}{0pt}}
{\end{enumerate}}

\newcounter{pagenum}
\setcounter{pagenum}{1}
\newcommand{\pageheader}[1]{
\clearpage\vspace*{-0.4in}\noindent{\large\bf{Page \arabic{pagenum}: {#1}}}
\addtocounter{pagenum}{1}
\cfoot{}
}

\newcounter{quesnum}
\setcounter{quesnum}{1}
\newcommand{\question}[2][??]{
\begin{list}{\labelitemi}{\leftmargin=2em}
\item [\arabic{quesnum}.] {} {#2}
\end{list}
\addtocounter{quesnum}{1}
}


\definecolor{red}{rgb}{1.0,0.0,0.0}
\newcommand{\answer}[2][??]{
\ifthenelse{\boolean{solution}}{
\color{red} #2 \color{black}}
{\vspace*{#1}}
}

\definecolor{blue}{rgb}{0.0,0.0,1.0}

\begin{document}

\section*{\homework}


\question[3]{
For each of the following claims, state whether it is true or false and then prove your assertion.
}

\begin{itemize}
	\item All finite sets have an \emph{injection} to $\mathbb{N}$
	\item All finite sets have a \emph{surjection} to $\mathbb{N}$
	\item If $A$ is a countably infinite set (i.e., $|A|=|\mathbb{N}|$) and $B$ is a also a countably infinite set (i.e., $|B| = |\mathbb{N}|$), then $A \cup B$ is also countable.
	\item If $A$ is countably infinite and $B$ is uncountably infinite, then $A \cup B$ is countable.
	\item If $A$ is countably infinite and $B$ is uncountably infinite, then $A \cap B$ is countable.
\end{itemize}

\vspace{12pt}

\question[3]{
Consider the formal descriptions of each set below. For each, write a short informal English description of each set.
}

\begin{itemize}
	\item $\{n | \exists_{m \in \mathbb{N}} : n=2m\}$ 
	\item $\{n | \exists_{m \in \mathbb{N}} : n=2m \wedge \exists_{k \in \mathbb{N}} : n=3k \}$ 
	\item $\{ w \ | \ w \in \{0,1\}^* \wedge w=w^R \}$ \emph{**Note that $w^R$ is the reverse string of $w$ (e.g., 001 becomes 100)}
	\item $\{ n | n \in \mathbb{Z} \wedge n=n+1 \}$
\end{itemize}

\vspace{12pt}



\question[3]{
Consider a square grid with length and width $n$. The bottom left corner is considered position $(0,0)$ and the upper right corner is position $(n,n)$ \emph{(*Note that the first item in the tuple is the square along the horizontal axis and the second element is the index along the vertical axis)}. You can see an example grid below.\\
\\
Our goal is to find all the unique ways a robot starting at cell $(0,0)$ can reach cell $(n,n)$ by only moving up, down, left, right on the grid on each move. We would like you to do two things:

\begin{enumerate}
	\item In your own words, argue why the given set (the set of possible paths to the goal) is infinite (as opposed to finite).
	\item Show that the set of unique paths the robot can take to reach position $(n,n)$ is \emph{countably infinite}. \emph{Hint: Try showing that a superset of this one is countably infinite}.
\end{enumerate}
}

\includegraphics[scale=0.4]{img/lattice}

\vspace{12pt}

\question[3]{
Use a proof by diagonalization to show that the following set is uncountable:\\
\\
$F= \mathcal{P}(\mathbb{N}) $
\\
\\
In other words, prove that the power set of the natural numbers (the set of all subsets of the natural numbers) is uncountable.
}

\vspace{12pt}

\question[3]{
Suppose I build a new computing machine that can be programmed to recognize a lot of different functions! It is called the \emph{Flogrammable Device}. In order to program this machine, you can type out your program on a \emph{tape}, but this tape can only hold ten characters. Each character is from the alphabet $\Sigma=\{a,b,c,d,0,1\}$ and any combination of these 10 characters is a valid program. More precisely, a program is a String $P = p_1p_2...p_{10} \ | \ \forall_i \ p_i \in \Sigma$. You CANNOT have fewer than 10 characters or the code will not compile.\\
\\
Now suppose that you read online that somehow, there are 100,000,000 important functions that need to be computed by the \emph{Flogrammable Device} for it to cover all important functionality. Can we program each of the 100,000,000 functions on this machine? How do you know or not?
}



\end{document}
