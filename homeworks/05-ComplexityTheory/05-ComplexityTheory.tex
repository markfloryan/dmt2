\documentclass[12pt]{article}
\usepackage[top=1in,bottom=1in,left=0.75in,right=0.75in,centering]{geometry}
\usepackage{fancyhdr}
\usepackage{epsfig}
\usepackage[pdfborder={0 0 0}]{hyperref}
\usepackage{palatino}
\usepackage{wrapfig}
\usepackage{lastpage}
\usepackage{color}
\usepackage{ifthen}
\usepackage[table]{xcolor}
\usepackage{graphicx,type1cm,eso-pic,color}
\usepackage{hyperref}
\usepackage{amsmath}
\usepackage{amsfonts}
\usepackage{wasysym}



\hypersetup{
    colorlinks=true,
    linkcolor=blue,
    filecolor=magenta,      
    urlcolor=cyan,
    pdftitle={CS3120 - Homework 5},
    pdfpagemode=FullScreen,
    }

\urlstyle{same}

\def\course{CS 3120: Discrete Math and Theory II}
\def\homework{Complexity Theory}
\def\semester{Spring 2025}

\newboolean{solution}
\setboolean{solution}{false}

% add watermark if it's a solution exam
% see http://jeanmartina.blogspot.com/2008/07/latex-goodie-how-to-watermark-things-in.html
\makeatletter
\AddToShipoutPicture{%
\setlength{\@tempdimb}{.5\paperwidth}%
\setlength{\@tempdimc}{.5\paperheight}%
\setlength{\unitlength}{1pt}%
\put(\strip@pt\@tempdimb,\strip@pt\@tempdimc){%
\ifthenelse{\boolean{solution}}{
\makebox(0,0){\rotatebox{45}{\textcolor[gray]{0.95}%
{\fontsize{5cm}{3cm}\selectfont{\textsf{Solution}}}}}%
}{}
}}
\makeatother

\pagestyle{fancy}

\fancyhf{}
\lhead{\course}
\chead{Page \thepage\ of \pageref{LastPage}}
\rhead{\semester}
%\cfoot{\Large (the bubble footer is automatically inserted into this space)}

\setlength{\headheight}{14.5pt}

\newenvironment{itemlist}{
\begin{itemize}
\setlength{\itemsep}{0pt}
\setlength{\parskip}{0pt}}
{\end{itemize}}

\newenvironment{numlist}{
\begin{enumerate}
\setlength{\itemsep}{0pt}
\setlength{\parskip}{0pt}}
{\end{enumerate}}

\newcounter{pagenum}
\setcounter{pagenum}{1}
\newcommand{\pageheader}[1]{
\clearpage\vspace*{-0.4in}\noindent{\large\bf{Page \arabic{pagenum}: {#1}}}
\addtocounter{pagenum}{1}
\cfoot{}
}

\newcounter{quesnum}
\setcounter{quesnum}{1}
\newcommand{\question}[2][??]{
\begin{list}{\labelitemi}{\leftmargin=2em}
\item [\arabic{quesnum}.] {} {#2}
\end{list}
\addtocounter{quesnum}{1}
}


\definecolor{red}{rgb}{1.0,0.0,0.0}
\newcommand{\answer}[2][??]{
\ifthenelse{\boolean{solution}}{
\color{red} #2 \color{black}}
{\vspace*{#1}}
}

\definecolor{blue}{rgb}{0.0,0.0,1.0}

\begin{document}

\section*{\homework}



\question[3]{
Show that if $P = NP$, then every language $A \in P$, except $A = \emptyset$ and $A = \Sigma^*$ is \emph{NP-Complete}.
}

\vspace{12pt}



\question[3]{
\emph{PSPACE} is the complexity class containing all problems that can be solved using a \emph{Deterministic Turing Machine} that uses at most a polynomial amount of additional space (on the tape). Prove that $P \subseteq \textit{PSPACE}$. (\emph{Hint: Think about a generic problem in P, and the DTM that decides it. Try to use more than a polynomial space with this machine given the allotted time.})
}


\vspace{12pt}

\question[3]{
Recall the \emph{knapsack problem}: Given a list of item values $V = \{v_1,v_2,...,v_n\}$, their respective weights $W = \{w_1,w_2,...,w_n\}$, a target value $k \in \mathbb{N}$, and a knapsack capacity $C \in \mathbb{N}$, does there exist a subset of items such that the sum of the values of the items is at least $k$ and the sum of the weight of the items is less than or equal to $C$. In other words, choose items $I$ such that $\sum_{i \in I}(v_i) \geq k$ and $\sum_{i \in I}(w_i) \leq C$. Show that the \emph{knapsack problem} is NP-Complete. You can use any other known NP-Complete problem for your reduction (you might want to look up other NP-Complete problems for this, but you don't have to!).
}

\vspace{12pt}

\question[3]{
You are working for \emph{Shoogle} (A data company that definitely does not exist) and you are given the following task from your boss: \emph{Shoogle} is interested in investing in their employees by offering a large amount of professional development. They have collected a series of \emph{workshops} that each cover a subset of \emph{topics}. For example, workshop 1 might cover machine learning and databases, workshop 2 might cover machine learning and data mining, and workshop 3 might cover advanced data structures (note that each workshop can cover multiple topics and there can be overlap across workshops). In addition, \emph{Shoogle} wants to make sure that over the course of the professional development workshops, at least one topic in each sub-area of interest is represented. For example, they might want to make sure there is at least one AI workshop (machine learning or reinforcement learning or neural networks, etc.), at least one systems workshop (databases or cloud computing), etc.\\
\\
So...the problem is this: Given the availability of the professional development workshop, what topics each covers, and which areas need to be covered, can you devise a schedule over multiple weeks that ensures that at least one topic in every area is covered by at least one workshop?\\
\\
Let's formalize this a bit: The input contains multiple items:

\begin{itemize}
    \item $T$: A list of individual topics that can be covered (e.g., machine learning).
    \item $W=\{w_1,w_2,...,w_m | w_i \subseteq T\}$: A list of workshops, each of which is a subset of the topics $T$ that will be covered in that workshop. There are $m$ workshops total.
    
    \item $S=\{s_1,s_2,...,s_n | s_i \subseteq W\}$: The schedule availability for each week. Each $s_i$ is the subset of workshops that can be scheduled in week $i$. There are $n$ weeks. You can only schedule ONE workshop per week.
    \item $A= \{a_1,a_2,...,a_p | a_i \subseteq T\}$: The subject areas that need to be covered. Each area is a list of Topics in that area.
\end{itemize}

You want to output Yes if there is some schedule of workshops over the $n$ weeks such that every area has at least one topic in $T$ that is covered in at least one workshop.\\
\\
This problem is \emph{NP-Complete}. Prove it! For your reduction, use \emph{3-SAT}.
}

\vspace{12pt}



\end{document}
