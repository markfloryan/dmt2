\documentclass[12pt]{article}
\usepackage[top=1in,bottom=1in,left=0.75in,right=0.75in,centering]{geometry}
\usepackage{fancyhdr}
\usepackage{epsfig}
\usepackage[pdfborder={0 0 0}]{hyperref}
\usepackage{palatino}
\usepackage{wrapfig}
\usepackage{lastpage}
\usepackage{color}
\usepackage{ifthen}
\usepackage[table]{xcolor}
\usepackage{graphicx,type1cm,eso-pic,color}
\usepackage{hyperref}
\usepackage{amsmath}
\usepackage{wasysym}


\hypersetup{
    colorlinks=true,
    linkcolor=blue,
    filecolor=magenta,      
    urlcolor=cyan,
    pdftitle={Overleaf Example},
    pdfpagemode=FullScreen,
    }

\urlstyle{same}

\def\course{CS 3120: Discrete Math and Theory II}
\def\homework{Turing Machines and Decidability}
\def\semester{Fall 2023}

\newboolean{solution}
\setboolean{solution}{false}

% add watermark if it's a solution exam
% see http://jeanmartina.blogspot.com/2008/07/latex-goodie-how-to-watermark-things-in.html
\makeatletter
\AddToShipoutPicture{%
\setlength{\@tempdimb}{.5\paperwidth}%
\setlength{\@tempdimc}{.5\paperheight}%
\setlength{\unitlength}{1pt}%
\put(\strip@pt\@tempdimb,\strip@pt\@tempdimc){%
\ifthenelse{\boolean{solution}}{
\makebox(0,0){\rotatebox{45}{\textcolor[gray]{0.95}%
{\fontsize{5cm}{3cm}\selectfont{\textsf{Solution}}}}}%
}{}
}}
\makeatother

\pagestyle{fancy}

\fancyhf{}
\lhead{\course}
\chead{Page \thepage\ of \pageref{LastPage}}
\rhead{\semester}
%\cfoot{\Large (the bubble footer is automatically inserted into this space)}

\setlength{\headheight}{14.5pt}

\newenvironment{itemlist}{
\begin{itemize}
\setlength{\itemsep}{0pt}
\setlength{\parskip}{0pt}}
{\end{itemize}}

\newenvironment{numlist}{
\begin{enumerate}
\setlength{\itemsep}{0pt}
\setlength{\parskip}{0pt}}
{\end{enumerate}}

\newcounter{pagenum}
\setcounter{pagenum}{1}
\newcommand{\pageheader}[1]{
\clearpage\vspace*{-0.4in}\noindent{\large\bf{Page \arabic{pagenum}: {#1}}}
\addtocounter{pagenum}{1}
\cfoot{}
}

\newcounter{quesnum}
\setcounter{quesnum}{1}
\newcommand{\question}[2][??]{
\begin{list}{\labelitemi}{\leftmargin=2em}
\item [\arabic{quesnum}.] {} {#2}
\end{list}
\addtocounter{quesnum}{1}
}


\definecolor{red}{rgb}{1.0,0.0,0.0}
\newcommand{\answer}[2][??]{
\ifthenelse{\boolean{solution}}{
\color{red} #2 \color{black}}
{\vspace*{#1}}
}

\definecolor{blue}{rgb}{0.0,0.0,1.0}

\begin{document}

\section*{\homework}


\question[3]{
Give implementation level descriptions for Turing Machines that decides the following two languages.
}

\begin{itemize}
	\item $\{w | w \text{contains an equal number of 0s and 1s} \}$
	\item $\{w | w \text{contains an twice as many 0s as 1s} \}$
\end{itemize}


\vspace{12pt}

\question[3]{
A \emph{2-PDA} is a pushdown automata that has two stacks instead of just one. Prove that a 2-PDA is more powerful than a traditional PDA with one stack. \emph{Hint: Show how to simulate a Turing Machine's tape with the 2-PDA.}
}


\vspace{12pt}

\question[3]{
For this question, you will do three separate proofs. Prove that the class of \emph{Decidable Languages} is closed under union, concatenation, and star.
}

\vspace{12pt}

\question[3]{
Prove the following claim: Let $C$ be a language. Prove that $C$ is Turing-recognizable if and only if a decidable language $D$ exists such that $C=\{x | \exists y ((x,y) \in D)\}$
}




\vspace{12pt}

\question[3]{
In the \emph{Harry Potter} books / movies, the Weasley family owns a magical clock that can tell when the Weasley children are (among other things) in \emph{mortal danger!}. See \href{https://harrypotter.fandom.com/wiki/Weasley_Clock}{this link} for details. Is it theoretically possible to build a clock with this functionality? In this problem you will show that building this function is undecidable.\\
\\
First, let's change the function of the clock to make this question less grim. Suppose instead of detecting whether Ron Weasley is in mortal danger, the clock detects whether Ron Weasley is in danger of getting his hair cut off. If the clock registers this danger setting, then Ron is going to get his hair cut off unless influenced by an outside force (the clock just knows!!). If the clock does not register this danger setting, then Ron's hair is safe from the clippers!\\
\\
Your question is this: Show that this clock's functionality is undecidable through reduction. Specifically, show that if this clock existed, you could construct a machine that decides the halting problem. Describe how to construct this machine. Your solution will likely require Turing Machines, physical components, razors, and Ron Weasley himself!
}

\end{document}
